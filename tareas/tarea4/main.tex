\documentclass[letterpaper,12pt]{memoir}
\usepackage[utf8]{inputenc}
\usepackage[spanish,es-tabla]{babel}
\usepackage{amsfonts}
\usepackage{mathptmx}
\usepackage[T1]{fontenc}
\usepackage[margin=1.3in]{geometry}
\usepackage{amsthm}
\usepackage{marvosym}
\usepackage{bm}
\usepackage{tikz}
\usepackage[tableaux]{prooftrees}

\linespread{1.3}


\usetikzlibrary{automata, positioning, arrows, fit}
\tikzset{
  ->,
  >=stealth',
  node distance=2cm,
  every state/.style={thick},
  initial text=$ $,
}

\renewcommand\qedsymbol{\Squarepipe}

\theoremstyle{definition}
\newtheorem{definition}{Definición}[section]
\newtheorem*{thm}{Teorema}


\setlength\parindent{0pt}

\newcounter{paragraphnumber}
\newcommand{\para}{%
  \vspace{10pt}\noindent{\bfseries\refstepcounter{paragraphnumber}\theparagraphnumber.\quad}%
}

\setsecheadstyle{\large\bfseries}
\setsubsecheadstyle{\bfseries}

\setlength\parindent{0pt}

\pagenumbering{gobble}

%\usepackage[margin=1in]{geometry}

\usepackage{enumitem}
\setlist{nosep}

\usepackage{xcolor}

\usepackage{hyperref}
\hypersetup{
  colorlinks,
  linkcolor={red!50!black},
  citecolor={blue!50!black},
  urlcolor={green!50!black}
}

\usepackage{amssymb}
\usepackage{amsmath}

\begin{document}

\begin{center}
  {\large Lógica Computacional}\\
  \vspace{0.2cm}
  {\large\bfseries Tarea 4}\\
  \vspace{0.2cm}
  {\large PCIC - UNAM}\\
  \vspace{0.5cm}
  {\itshape 14 de abril de 2020}\\
  \vspace{0.5cm}
  Diego de Jesús Isla López\\
  (\href{mailto:dislalopez@gmail.com}{\itshape dislalopez@gmail.com})\\
  (\href{mailto:diego.isla@comunidad.unam.mx}{\itshape diego.isla@comunidad.unam.mx})\\
\end{center}


\section*{Problema 7.9}

Demostrar que \( (\forall xp(x) \rightarrow \forall xq(x)) \rightarrow \forall x (p(x) \rightarrow q(x))  \) no es válida construyendo un tableau semántico para su negación.

\begin{proof}
  Por tableau semántico:\\
  \noindent

  \begin{tableau}
    {
      line no sep= 1.5cm,
      just sep= 1.5cm,
      for tree={s sep'=10mm},
      just refs right, % Set where crossreferences go
    }
    [\neg((\forall xp(x) \rightarrow \forall xq(x)) \rightarrow \forall x (p(x) \rightarrow q(x))), just={Negación}, name=Prem
    [(\forall xp(x) \rightarrow \forall xq(x)) \texttt{,} \neg(\forall x (p(x) \rightarrow q(x)), just={\(\alpha \rightarrow\) }, name=NegConc
    [(\forall xp(x) \rightarrow \forall xq(x)) \texttt{,} \neg(p(a_1)\rightarrow q(a_1)), just={\(\delta \forall\)}
    [\neg \forall xp(x) \texttt{,} \neg(p(a_1) \rightarrow q(a_1)), just={\(\alpha \rightarrow\)},name=Bertie
    [\neg p(a_2) \texttt{,} p(a_1) \texttt{,} \neg q(a_1), just={\(\delta \forall\)}
    ]
    ]
    [
      \forall xq(x) \texttt{,} \neg(p(a_1) \rightarrow q(a_1), just={}
      [\forall xq(x) \texttt{,} q(a_1) \texttt{,} p(a_1) \texttt{,} \neg q(a_1) , close, just={\(\gamma \forall\)}
    ]
    ]
    ]
    ]
    ]
  \end{tableau}
  \newline

  El tableau tiene una rama abierta por lo que no es cerrado, por lo que se sigue que la fórmula no es válida.
\end{proof}


\section*{Problema 8.3}

Demostrar que los axiomas 4 y 5 son válidos.\\

\textbf{Axioma 4.} \(\vdash \forall xA(X) \rightarrow A(a)\)

\begin{proof}
Por tableau semántico en la negación de la fórmula:\\
\begin{tableau}
  {
    line no sep= 1.5cm,
    just sep= 1.5cm,
    for tree={s sep'=10mm},
    just refs right, % Set where crossreferences go
  }
  [
    \neg(\forall xA(x) \rightarrow A(a)), just={Negación}
    [\forall xA(x) \texttt{,} \neg A(a), just={\(\alpha \rightarrow\)}
      [\forall xA(x) \texttt{,} A(a) \texttt{,} \neg A(a), close, just={\(\gamma \forall\)}]
    ]
  ]
\end{tableau}
\newline
El tableau es cerrado, por lo que la fórmula es válida.
\end{proof}

\textbf{Axioma 5.} \(\vdash \forall x(A \rightarrow B(x)) \rightarrow (A \rightarrow \forall xB(x))\)

\begin{proof}
  Por tableau semántico en la negación de la fórmula:\\
  \newline
  \begin{tableau}
    {
      line no sep= 1.5cm,
      just sep= 1.5cm,
      for tree={s sep'=10mm},
      just refs right, % Set where crossreferences go
    }
    [
     \neg(\forall x(A \rightarrow B(x)) \rightarrow (A \rightarrow \forall xB(x))),just={Negación}
    [\forall x(A \rightarrow B(x)) \texttt{,} \neg(A \rightarrow \forall xB(x)),just={\(\alpha \rightarrow\)}
    [\forall x(A \rightarrow B(x)) \texttt{,} A \texttt{,} \neg \forall xB(x),just={\(\alpha \rightarrow\)}
      [\forall x(A \rightarrow B(x)) \texttt{,} A \texttt{,} \neg B(a_1),just={\(\delta \forall\)}
        [(A \rightarrow \forall xB(x)) \texttt{,} A \texttt{,} \neg B(a_1),just={Distributividad de \(\forall\)}
        [\neg A \texttt{,} A \texttt{,} \neg A(a_1), close,just={\(\beta \rightarrow\)}]
        [\forall xB(x) \texttt{,} A \texttt{,} \neg B(a_1),just={\(\beta \rightarrow\)}
        [\forall xB(x) \texttt{,} B(a_1) \texttt{,} A \texttt{,} \neg B(a_1),close, just={\(\gamma \forall\)}]
        ]
        ]
      ]
    ]
    ]
    ]
  \end{tableau}
  \newline
  El tableau resulta cerrado, por lo que la fórmula es válida.
\end{proof}


\section*{Problema 8.8}

Demostrar el teorema 8.19 en \(H\).\\
\newline

\textbf{Teorema 8.19} Sea \(A\) una fórmula que no tiene \(x\) como variable libre:
\begin{align}
  \vdash \forall x(A \rightarrow B(x)) & \leftrightarrow (A \rightarrow \forall xB(x)),\\
  \vdash \exists x(A \rightarrow B(x)) & \leftrightarrow  (A \rightarrow \exists xB(x))
\end{align}



\begin{proof}
  Para (1) tenemos: \\

  \begin{itemize}
    \item \(\forall (A \rightarrow B(x)) \rightarrow (A \rightarrow \forall xB(x))\)
    \begin{align*}
      1. & & \forall (A \rightarrow B(x)) &\vdash \forall (A \rightarrow B(x)) & & \texttt{Sup.}  \\
      2. & & \forall (A \rightarrow B(x)) &\vdash \forall (\neg A \lor B(x)) & & \texttt{Def. implicación}\\
      3. & &  \forall (A \rightarrow B(x)) &\vdash \neg A \lor \forall xB(x) & & \texttt{Distributividad} \\
      4. & &  \forall (A \rightarrow B(x)) &\vdash A \rightarrow \forall xB(x) & & \texttt{Def. implicación} \\
      5. & & \forall (A \rightarrow B(x)) &\rightarrow (A \rightarrow \forall xB(x)) & & \texttt{Deducción} 
    \end{align*}
    \item \( (A \rightarrow \forall xB(x)) \rightarrow \forall (A \rightarrow B(x)) \)
    \begin{align*}
      1. & & A \rightarrow \forall xB(x) &\vdash A \rightarrow \forall xB(x) & &  \texttt{Sup.}  \\
      2. & & A \rightarrow \forall xB(x) &\vdash \neg A \lor \forall xB(c)  & & \texttt{Def. implicación}\\
      3. & & A \rightarrow \forall xB(x) &\vdash \forall x(\neg A \lor B(x)) & & \texttt{Distributividad} \\
      4. & & A \rightarrow \forall xB(x) &\vdash \forall x(A \rightarrow B(x)) & & \texttt{Def. implicación} \\
      5. & & (A \rightarrow \forall xB(x)) &\rightarrow \forall x(A \rightarrow B(x)) & & \texttt{Deducción} 
    \end{align*}
  \end{itemize}


  Para (2) tenemos: \\

  \begin{itemize}
    \item \( \exists (A \rightarrow B(x)) \rightarrow (A \rightarrow \exists xB(x)) \)
    \begin{align*}
      1. & & \exists (A \rightarrow B(x)) &\vdash \exists (A \rightarrow B(x)) & & \texttt{Sup.} \\
      2. & & \exists (A \rightarrow B(x)) &\vdash A \rightarrow B(a) & & \texttt{Regla C} \\
      3. & &  \exists (A \rightarrow B(x)) &\vdash A \rightarrow \exists xB(x) & & \texttt{Teorema 8.14} \\
      4. & & \exists (A \rightarrow B(x)) &\rightarrow (A \rightarrow \exists xB(x)) & & \texttt{Deduccion}
    \end{align*}
    \item \( (A \rightarrow \exists xB(x)) \rightarrow \exists (A \rightarrow B(x)) \)
    \begin{align*}
      1. & & (A \rightarrow \exists xB(x)) &\vdash (A \rightarrow \exists xB(x)) & & \texttt{Sup.} \\
      2. & & (A \rightarrow \exists xB(x)) &\vdash \neg A \lor \exists xB(c)  & & \texttt{Def. implicación}\\
      3. & &  (A \rightarrow \exists xB(x)) &\vdash \exists x(\neg A \lor B(x)) & & \texttt{Distributividad} \\
      4. & & (A \rightarrow \exists xB(x)) &\vdash \exists x(A \rightarrow B(x)) & & \texttt{Def. implicación} \\
      5. & & (A \rightarrow \exists xB(x)) &\rightarrow \exists x(A \rightarrow B(x)) & & \texttt{Deducción} 
    \end{align*}



  \end{itemize}
\end{proof}

\end{document}