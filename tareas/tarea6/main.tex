\documentclass[letterpaper,12pt]{memoir}
\usepackage[utf8]{inputenc}
\usepackage[spanish,es-tabla]{babel}
\usepackage{amsfonts}
\usepackage{mathptmx}
\usepackage[T1]{fontenc}
\usepackage[margin=1.3in]{geometry}
\usepackage{amsthm}
\usepackage{marvosym}
\usepackage{bm}
\usepackage{tikz}
\usepackage[tableaux]{prooftrees}

\linespread{1.3}


\usetikzlibrary{automata, positioning, arrows, fit}
\tikzset{
  ->,
  >=stealth',
  node distance=2cm,
  every state/.style={thick},
  initial text=$ $,
}

\renewcommand\qedsymbol{\Squarepipe}

\theoremstyle{definition}
\newtheorem{definition}{Definición}[section]
\newtheorem*{thm}{Teorema}
\newtheorem{solution}{Solución}


\setlength\parindent{0pt}

\newcounter{paragraphnumber}
\newcommand{\para}{%
  \vspace{10pt}\noindent{\bfseries\refstepcounter{paragraphnumber}\theparagraphnumber.\quad}%
}

\setsecheadstyle{\large\bfseries}
\setsubsecheadstyle{\bfseries}

\setlength\parindent{0pt}

\pagenumbering{gobble}

%\usepackage[margin=1in]{geometry}

\usepackage{enumitem}
\setlist{nosep}

\usepackage{xcolor}

\usepackage{hyperref}
\hypersetup{
  colorlinks,
  linkcolor={red!50!black},
  citecolor={blue!50!black},
  urlcolor={green!50!black}
}

\usepackage{amssymb}
\usepackage{amsmath}

\begin{document}

\begin{center}
  {\large Lógica Computacional}\\
  \vspace{0.2cm}
  {\large\bfseries Tarea 6}\\
  \vspace{0.2cm}
  {\large PCIC - UNAM}\\
  \vspace{0.5cm}
  {\itshape 21 de mayo de 2020}\\
  \vspace{0.5cm}
  Diego de Jesús Isla López\\
  (\href{mailto:dislalopez@gmail.com}{\itshape dislalopez@gmail.com})\\
  (\href{mailto:diego.isla@comunidad.unam.mx}{\itshape diego.isla@comunidad.unam.mx})\\
\end{center}


\section*{Problema 15.2}

Sean \verb|S1|: \verb|x = x+y| y  \verb|S2|: \verb|y = x*y|, ¿cuál es la precondición más débil \(wp(S1,S2,x<y)\)?

\begin{solution}

Primero obtenemos \(wp(S2, x<y)\):

\begin{align*}
wp(\texttt{y=x*y},x<y) &= x < x*y
\end{align*}

Usando el resultado anterior\\
	\begin{align*}
	wp(\texttt{x=x+y},\texttt{y=x*y},x<y)  &= wp(\texttt{x=x+y}, x<x*y)\\
										  &= x<x*y[x\leftarrow x+y]\\
										  &= x+y<(x+y)*y\\
	\end{align*}
\end{solution}

\section*{Problema 15.7}
\begin{proof}
	Tomamos como invariante de bucle \(\{(0 \leq x^2 \leq a) \land y = (x+1)^2\}\).
	
	Para demostrar \((0 \leq x^2 \leq a)\) tenemos que:
	
	\begin{align*}
	x'^2 &= (x+1)^2 = y 
	\end{align*}
	
	Esto es, que el valor de \(x'\) siempre será igual a \(y\) antes de que esta variable sea modificada.
	
	Para demostrar \(y = (x+1)^2\) tenemos:
	
	\begin{align*}
	x' &= x + 1\\
	y' &= y + 2x' + 1\\
	y' &= (x+1)^2 + 2(x+1) + 1\\
	y' &= x^2 + 4x + 4\\
	y' &= (x+2)^2\\
	y' &= (x'+1)^2\\
	\end{align*}
\end{proof}




\section*{Problema 15.10}

\begin{proof}
	Tomamos la invariante \(z \cdot x^y = a^b\). En el caso en el que \(y\) sea par, podemos decir sin pérdida de generalidad que \(y = 2k\) para alguna \(k\). Entonces \(z\cdot x^{2k} = z \cdot (x^2)^k \). En el caso en el que \(y\) es impar, tenemos \(z \cdot x^y = z \cdot (x \cdot x^{y-1}) = (z \cdot x) \cdot x^{y-1} \). Al final del ciclo, se tiene \(y = 0\) y por lo tanto \(z \cdot x^0 = z = a^b\).
	
\end{proof}


 

\end{document}