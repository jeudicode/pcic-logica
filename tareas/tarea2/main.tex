\documentclass[letterpaper,12pt]{article}
\usepackage[utf8]{inputenc}
\usepackage[spanish,es-tabla]{babel}
\usepackage{amsfonts}
\usepackage{mathptmx}
\usepackage[T1]{fontenc}
\usepackage[margin=1.3in]{geometry}
\usepackage{amsthm}
\usepackage{marvosym}
\usepackage{bm}
\usepackage{tikz}
\usepackage[tableaux]{prooftrees}


\linespread{1.3}


\usetikzlibrary{automata, positioning, arrows, fit}
\tikzset{
  ->,
  >=stealth',
  node distance=2cm,
  every state/.style={thick},
  initial text=$ $,
}

\renewcommand\qedsymbol{\Squarepipe}

\theoremstyle{definition}
\newtheorem{definition}{Definición}[section]
\newtheorem*{thm}{Teorema}


\setlength\parindent{0pt}

\newcounter{paragraphnumber}
\newcommand{\para}{%
  \vspace{10pt}\noindent{\bfseries\refstepcounter{paragraphnumber}\theparagraphnumber.\quad}%
}

%\setsecheadstyle{\large\bfseries}
%\setsubsecheadstyle{\bfseries}

\setlength\parindent{0pt}

\pagenumbering{gobble}

%\usepackage[margin=1in]{geometry}

\usepackage{enumitem}
\setlist{nosep}

\usepackage{xcolor}

\usepackage{hyperref}
\hypersetup{
  colorlinks,
  linkcolor={red!50!black},
  citecolor={blue!50!black},
  urlcolor={green!50!black}
}

\usepackage{amssymb}
\usepackage{amsmath}

\begin{document}

\begin{center}
  {\large Lógica Computacional}\\
  \vspace{0.2cm}
  {\large\bfseries Tarea 2}\\
  \vspace{0.2cm}
  {\large PCIC - UNAM}\\
  \vspace{0.5cm}
  {\itshape 12 de marzo de 2020}\\
  \vspace{0.5cm}
  Diego de Jesús Isla López\\
  (\href{mailto:dislalopez@gmail.com}{\itshape dislalopez@gmail.com})\\
  (\href{mailto:diego.isla@comunidad.unam.mx}{\itshape diego.isla@comunidad.unam.mx})\\
\end{center}


% \section*{Problema 4.3}

% Simplificar los siguientes conjuntos de cláusulas. Esto es, para cada conjunto \(S\), encontrar un conjunto más simple \(S'\) tal que  \(S'\) es satisfactible sii  \(S\) es satisfactible.

% \begin{enumerate}
%   \item \(p\bar{q}, q\bar{r}, rs, p\bar{s}\):
%     \begin{align*}
%       &=  p \bar{r}, rs, p \bar{s}\\
%       &=  ps, p \bar{s}\\
%       &=  p
%     \end{align*}
%   \item \(pqr, \bar{q}, p\bar{r}s, qs, p\bar{s}\):
%     \begin{align*}
%       &=  pr, p\bar{r}s, qs, p\bar{s}\\
%       &=  ps, qs, p\bar{s}\\
%       &=  p, qs
%     \end{align*}
%   \item \(pqrs, \bar{q}rs, \bar{p}rs, qs, \bar{p}s\):
%     \begin{align*}
%       &=  prs, \bar{p}rs, qs, \bar{p}s\\
%       &= rs, qs, \bar{p}s
%     \end{align*}
%   \item \(pq, qrs, \bar{p}\bar{q}rs, \bar{r}, q\):
%   \begin{align*}
%     &= pq, qrs, \bar{p}rs, \bar{r}\\
%     &=  pq, qrs, \bar{p}s\\
%     &=  qs, qrs
%   \end{align*}
% \end{enumerate}


% \section*{Problema 4.4}

% Dado el conjunto de cláusulas \(\{\bar{p}\bar{q}r, pr, qr, \bar{r} \}\), construir dos refutaciones: una resolviendo las literales en el orden \(\{p, q, r\}\) y otra en el orden \(\{r, q, p\}\).\\

% \begin{enumerate}
%   \item \(\{\bar{p}\bar{q}r (1), pr (2), qr (3), \bar{r} (4) \}\):
%     \begin{align*}
%       5. & \bar{q}r & 1,2\\
%       6. & r & 3,5\\
%       7. & \square & 4,6 
%     \end{align*}
%   \item \(\{\bar{r} (1), r\bar{q}\bar{p} (2), rp (3), rq(4)\}\):
%   \begin{align*}
%     5. & \bar{q}\bar{p} & 1,2\\
%     6. & p & 1,3\\
%     7. & q & 1,4\\
%     8. & \bar{p} & 5,7\\
%     9. & \square & 6,8
%   \end{align*}
% \end{enumerate}



% \section*{Problema 4.5}

% Transformar el conjunto de fórmulas \(\{p, p \rightarrow ((q \lor r)\land \neg(q \land r)), p \rightarrow ((s \lor t) \land \neg (s \land t)), s \rightarrow q, \neg r \rightarrow t, t \rightarrow s  \}\) a CNF y refutarla usando resolución.\\ 

% Haciendo la transformación, tenemos:

% \begin{equation*}
%   \{p(1), \bar{p}qr(2), \bar{p}\bar{q}\bar{r}(3), \bar{p}st(4), \bar{p}\bar{s}\bar{t}(5), \bar{s}q(6), rt(7), \bar{t}s(8) \}
% \end{equation*}

% Aplicando el algoritmo:\\

% \begin{align*}
%    9. & st & 1,4\\
%    10. & s & 8,9 \\
%    11. & q & 10,6\\
%    12. & qr & 1,2\\
%    13. & \bar{q}\bar{r} & 1,3\\
%    14. & \bar{s}\bar{t} & 1,5\\
%    15. & \bar{t} & 14,8\\
%    16. & r & 15,7\\
%    17. & \bar{q} & 16,13\\
%    18. & \square & 11,17
% \end{align*}


\section*{Problema 4.13}

Demostrar el teorema 4.13 sobre la correctitud del algoritmo CNF-a-3CNF.

\begin{thm}
  Sea \(A\) una fórmula en CNF y sea \(A'\) la fórmula en 3CNF construida a partir de \(A\). Entonces \(A\) es satisfactible sii \(A'\) es satisfactible. La longitud de \(A'\) es un polinomio en la longitud de \(A\).
\end{thm}


\begin{proof}
  Sea \(A\) un conjunto de literales tal que \(|A| = k\) para \(k >= 1\). La demostración procede por casos según la longitud de \(A\) y el resultado del algoritmo.

  \begin{itemize}
    \item \textbf{Caso 1 (\(k = 1\)): } Entonces, \(A = \{x_1\}\) y \(A' = \{ \{x_1, y, z\}, \{x_1, \bar{y}, z\}, \{x_1, y, \bar{z}\}, \{x_1, \bar{y}, \bar{z} \} \}\). Dado que \(A\) consta de una sola literal, si esta se satisface, entonces todas las cláusulas de \(A'\) se satisfacen.
    \item \textbf{Caso 2 (\(k = 2\)): } Entonces, \(A = \{x_1, x_2\}\) y \(A' = \{ \{x_1, x_2, z\}, \{x_1, x_2, \bar{z} \}\). Dado que \(A\) consta de dos literales, basta con que alguna de ellas se satisfaga para que las cláusulas de \(A'\) se satisfagan.
    \item \textbf{Caso 3 (\(k > 3\)): } Entonces, \(A = \{x_1, x_2, x_3, ..., x_k\}\). Tomemos \(x_m = 1\) para alguna \(m\) tal que \(2 < m < k - 1\). Tomemos \(z_i = 1\) para toda \(i \leq m - 2\) y \(z_j = 0\) para toda \(j \geq m - 1\). Sea \( A'_m \in A' \) la cláusula que contiene a \(x_m\). Todas las cláusulas a la izquierda de \( A'_m\) tendrán una tercera literal \(z_i = 1\); a su vez, todas las cláusulas a la derecha de \( A'_m\) tendrán una primera literal \(\bar{z_j} = 1\). De este modo, todas las cláusulas de \(A'\) se satisfacen.
  \end{itemize} 
\end{proof}



\end{document}