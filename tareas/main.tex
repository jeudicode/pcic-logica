\documentclass[letterpaper,12pt]{memoir}
\usepackage[utf8]{inputenc}
\usepackage[spanish,es-tabla]{babel}
\usepackage{amsfonts}

\usepackage{mathptmx}
\usepackage[T1]{fontenc}
\usepackage[margin=1.3in]{geometry}
\usepackage{amsthm}
\usepackage{marvosym}
\usepackage{bm}

\renewcommand\qedsymbol{\Squarepipe}

\theoremstyle{definition}
\newtheorem{definition}{Definición}[section]
\newtheorem*{thm}{Teorema}


\setlength\parindent{0pt}

\newcounter{paragraphnumber}
\newcommand{\para}{%
  \vspace{10pt}\noindent{\bfseries\refstepcounter{paragraphnumber}\theparagraphnumber.\quad}%
}

\setsecheadstyle{\large\bfseries}
\setsubsecheadstyle{\bfseries}

\setlength\parindent{0pt}

\pagenumbering{gobble}

\usepackage[margin=1in]{geometry}

\usepackage{enumitem}
\setlist{nosep}

\usepackage{xcolor}

\usepackage{hyperref}
\hypersetup{
  colorlinks,
  linkcolor={red!50!black},
  citecolor={blue!50!black},
  urlcolor={green!50!black}
}

\usepackage{amssymb}
\usepackage{amsmath}

\begin{document}

\begin{center}
  {\large Lógica Computacional}\\
  \vspace{0.2cm}
  {\large\bfseries Tarea 1}\\
  \vspace{0.2cm}
  {\large PCIC - UNAM}\\
  \vspace{0.5cm}
  {\itshape 11 de febrero de 2020}\\
  \vspace{0.5cm}
  Diego de Jesús Isla López\\
  (\href{mailto:dislalopez@gmail.com}{\itshape dislalopez@gmail.com})\\
  (\href{mailto:diego.isla@comunidad.unam.mx}{\itshape diego.isla@comunidad.unam.mx})\\
\end{center}

\begin{thm}
    Sea $I$ una interepretación. Entonces, $v_I(A_1 \wedge A_2 \wedge A_3 \wedge ... \wedge A_n) = 1$ sii $v_I(A_i) = 1$ para $i = 1 ... n$.
\end{thm}

\begin{proof}
    Supongamos que $A_i = 0$, para alguna $i = 1 ... n$. Sabemos por la tabla de verdad de la función $AND$ que $A \wedge B = 0$ si el valor de alguno de los operandos es $0$. De este modo, si hacemos la conjunción entre $A_i$ y $A_{i+1} ... A_n$, el resultado siempre será $0$.\\

    Supongamos ahora que $v_I(A_i) = 1$ para toda $i = 1 ... n$ y $v_I(A_1 \wedge A_2 \wedge A_3 \wedge ... \wedge A_n) = 1$. Supongamos para fines de contradicción que $v_I(A_1 \wedge A_2 \wedge A_3 \wedge ... \wedge A_n) = 0$. Esto implica que algún valor de $A_i = 0$, pero esto es una contradicción ya que sabemos que todos los $A_i = 1$.
\end{proof}


\end{document}