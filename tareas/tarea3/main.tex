\documentclass[letterpaper,12pt]{article}
\usepackage[utf8]{inputenc}
\usepackage[spanish,es-tabla]{babel}
\usepackage{amsfonts}
\usepackage{mathptmx}
\usepackage[T1]{fontenc}
\usepackage[margin=1.3in]{geometry}
\usepackage{amsthm}
\usepackage{marvosym}
\usepackage{bm}
\usepackage{tikz}
\usepackage[tableaux]{prooftrees}
\usepackage{soul}



\linespread{1.3}


\usetikzlibrary{automata, positioning, arrows, fit}
\tikzset{
  ->,
  >=stealth',
  node distance=2cm,
  every state/.style={thick},
  initial text=$ $,
}

\renewcommand\qedsymbol{\Squarepipe}

\theoremstyle{definition}
\newtheorem{definition}{Definición}[section]
\newtheorem*{thm}{Teorema}
\newtheorem*{solution}{Solución}



\setlength\parindent{0pt}

\newcounter{paragraphnumber}
\newcommand{\para}{%
  \vspace{10pt}\noindent{\bfseries\refstepcounter{paragraphnumber}\theparagraphnumber.\quad}%
}

%\setsecheadstyle{\large\bfseries}
%\setsubsecheadstyle{\bfseries}

\setlength\parindent{0pt}

\pagenumbering{gobble}

%\usepackage[margin=1in]{geometry}

\usepackage{enumitem}
\setlist{nosep}

\usepackage{xcolor}

\usepackage{hyperref}
\hypersetup{
  colorlinks,
  linkcolor={red!50!black},
  citecolor={blue!50!black},
  urlcolor={green!50!black}
}

\usepackage{amssymb}
\usepackage{amsmath}

\begin{document}

\begin{center}
  {\large Lógica Computacional}\\
  \vspace{0.2cm}
  {\large\bfseries Tarea 3}\\
  \vspace{0.2cm}
  {\large PCIC - UNAM}\\
  \vspace{0.5cm}
  {\itshape 31 de marzo de 2020}\\
  \vspace{0.5cm}
  Diego de Jesús Isla López\\
  (\href{mailto:dislalopez@gmail.com}{\itshape dislalopez@gmail.com})\\
  (\href{mailto:diego.isla@comunidad.unam.mx}{\itshape diego.isla@comunidad.unam.mx})\\
\end{center}


\section*{Problema 6.4}

Sea \(G\) una gráfica conexa. El problema de coloración de gráficas es decidir si uno de \(k\) colores \(\{c_1,\ldots, c_k\}\) puede ser asignado a cada vértice \(color(v_i) = c_j\) tal que \(color(v_{i1}) \neq color(v_{i2})\) si \((v_{i1}, v_{i1})\) es un arista en \(E\). Muestra cómo traducir el problema de coloración de gráficas para cada \(G\) a SAT. Usa el algoritmo DPLL para mostrar que \(K_{2,2}\) es 2-coloreable.

\begin{solution}
  Sea \(C = \{0,\ldots,k-1\}\) el conjunto de colores. El conjunto \(A\) de cláusulas constará de:
  \begin{itemize}
    \item Para cada \(v \in V\), existirá una cláusula \( (v_0, v_1,\ldots,v_{k-1}) \). Por lo tanto, habrá \(|V|\) cláusulas de este tipo.
    \item Para cada arista \( (u,v) \in E\), tendremos una cláusula \( (\neg u_c, \neg v_c) \) para cada \(c \in C \). De esta forma, habrá \( |V| \cdot |C| \) cláusulas de este tipo.
  \end{itemize}
\end{solution}

Sea una \(K_{2,2}\)-gráfica tal que \(V = \{a,b,c,d\}\) y \(E = \{(a,c),(a,d),(b,c),(b,d)\} \) y sea \(C = \{0,1\}\). Siguiendo el método, tenemos que el conjunto de cláusulas resulta:

\begin{equation}
  A = \{ a_0a_1, b_0b_1, c_0c_1, d_0d_1, \bar{a_0}\bar{c_0}, \bar{a_0}\bar{d_0}, \bar{a_1}\bar{c_1}, \bar{a_1}\bar{d_1}, \bar{b_0}\bar{c_0}, \bar{b_0}\bar{d_0}, \bar{b_1}\bar{c_1}, \bar{b_1}\bar{d_1}\}.
\end{equation}

Para mostrar el resultado del algoritmo, se usará la notación dimacs. De este modo, cada variable quedará representada de la siguiente manera: \(a_0 = 1\), \(a_1 = 2\), \(b_0 = 3\), \(b_1 = 4\), \(c_0 = 5\), \(c_1= 6\), \(d_0 = 7\), \(d_1 = 8\). En la tabla cada columna representa una iteración del algoritmo; las cláusulas tachadas son aquellas que contienen literales puras, por lo que se eliminan. Las cláulsulas marcadas con un asterisco son las que se eligen para continuar con la ejecución del algoritmo.\\

Dado que no hay cláusulas atómicas en el conjunto, se inicia el algoritmo usando \(a_0 = T\). De este modo, la ejecución del algoritmo queda como sigue:


  \begin{table}[h]
    \centering
    \begin{tabular}{ |c|c|c|c|c|c| } 
      \hline
      3 4   & 3 4        & 3 4   & 3 4        & 3 4     & 3 \\
      5 6   & 6*         & 7 8   & 8*         & \st{-2} &   \\
      7 8   & 7 8        & -7*   & -2         & \st{-2} &   \\
      -5*   & -7         & -2    & -2 -8      & -4*     &   \\
      -7    & -2 -6      & -2 -8 & \st{-3 -7} & -4      &   \\
      -2 -6 & -2 -8      & -3 -7 & -4         &         &   \\
      -2 -8 & \st{-3 -5} & -4    & -4 -8      &         &   \\
      -3 -5 & -3 -7      & -4 -8 &            &         &   \\
      -3 -7 & -4 -6      &       &            &         &   \\
      -4 -6 & -4 -8      &       &            &         &   \\
      -4 -8 &            &       &            &         &   \\
      \hline
    \end{tabular}
  \end{table}


Como se ve en la tabla, en la última iteración del algoritmo quedamos con solo una cláusula atómica, la correspondiente a \(b_0\). Por lo tanto, encontramos una solución para la fórmula, por lo que es satisfactible. Por lo tanto, \(K_{2,2}\) es 2-coloreable.

\end{document}