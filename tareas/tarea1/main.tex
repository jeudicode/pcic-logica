\documentclass[letterpaper,12pt]{memoir}
\usepackage[utf8]{inputenc}
\usepackage[spanish,es-tabla]{babel}
\usepackage{amsfonts}
\usepackage{mathptmx}
\usepackage[T1]{fontenc}
\usepackage[margin=1.3in]{geometry}
\usepackage{amsthm}
\usepackage{marvosym}
\usepackage{bm}
\usepackage{tikz}
\usepackage[tableaux]{prooftrees}

\linespread{1.3}


\usetikzlibrary{automata, positioning, arrows, fit}
\tikzset{
  ->,
  >=stealth',
  node distance=2cm,
  every state/.style={thick},
  initial text=$ $,
}

\renewcommand\qedsymbol{\Squarepipe}

\theoremstyle{definition}
\newtheorem{definition}{Definición}[section]
\newtheorem*{thm}{Teorema}


\setlength\parindent{0pt}

\newcounter{paragraphnumber}
\newcommand{\para}{%
  \vspace{10pt}\noindent{\bfseries\refstepcounter{paragraphnumber}\theparagraphnumber.\quad}%
}

\setsecheadstyle{\large\bfseries}
\setsubsecheadstyle{\bfseries}

\setlength\parindent{0pt}

\pagenumbering{gobble}

%\usepackage[margin=1in]{geometry}

\usepackage{enumitem}
\setlist{nosep}

\usepackage{xcolor}

\usepackage{hyperref}
\hypersetup{
  colorlinks,
  linkcolor={red!50!black},
  citecolor={blue!50!black},
  urlcolor={green!50!black}
}

\usepackage{amssymb}
\usepackage{amsmath}

\begin{document}

\begin{center}
  {\large Lógica Computacional}\\
  \vspace{0.2cm}
  {\large\bfseries Tarea 1}\\
  \vspace{0.2cm}
  {\large PCIC - UNAM}\\
  \vspace{0.5cm}
  {\itshape 25 de febrero de 2020}\\
  \vspace{0.5cm}
  Diego de Jesús Isla López\\
  (\href{mailto:dislalopez@gmail.com}{\itshape dislalopez@gmail.com})\\
  (\href{mailto:diego.isla@comunidad.unam.mx}{\itshape diego.isla@comunidad.unam.mx})\\
\end{center}


\section*{Problema 2.9}

Demostrar:

\begin{equation*}
  \models ((A \land B) \rightarrow C) \rightarrow ((A \rightarrow C) \lor (B \rightarrow C))  
\end{equation*}

Esta fórmula puede parecer extraña ya que puede malinterpretarse como decir que si \(C\) se sigue de \(A \land B\), entonces se sigue de \(A\) o de \(B\). Para clarificar esto, mostrar que:

\begin{equation*}
  \{A \land B \rightarrow C\}  \models (A \rightarrow C) \lor (B \rightarrow C),
\end{equation*}

pero:

\begin{align*}
  \{A \land B \rightarrow C\}  &\not\models A \rightarrow C, \\
  \{A \land B \rightarrow C\}  &\not\models B \rightarrow C,
\end{align*}

\begin{proof}
  Por tableau semántico en la negación de la fórmula:\\
  \noindent

  \begin{tableau}
    {
      line no sep= 1.5cm,
      just sep= 1.5cm,
      for tree={s sep'=10mm},
      just refs right, % Set where crossreferences go
    }
    [\neg(((A\land B)\rightarrow C) \rightarrow ((A \rightarrow C) \lor (B \rightarrow C))), just={Negación}, name=Prem
    [(A \land B) \rightarrow C \texttt{,} \neg((A \rightarrow C) \lor (B \rightarrow C )), just={\(\alpha \rightarrow\) }, name=NegConc
    [(A \land B) \rightarrow C \texttt{,} \neg(A \rightarrow C) \texttt{,} \neg(B \rightarrow C ), just={\(\alpha \rightarrow\)}
    [(A \land B) \rightarrow C \texttt{,} \neg(A \rightarrow C) \texttt{,} B \texttt{,} \neg C, just={\(\alpha \rightarrow\)},name=Alice
    [(A \land B) \rightarrow C \texttt{,} A \texttt{,} \neg C \texttt{,} B \texttt{,} \neg C, just={\(\alpha \rightarrow\)}
    [\neg(A \land B) \texttt{,} A \texttt{,} \neg C \texttt{,} B \texttt{,} \neg C, just={\(\beta \rightarrow\)},name=Bertie
    [\neg A \texttt{,} A \texttt{,} \neg C \texttt{,} B \texttt{,} \neg C, close, just={\(\beta \rightarrow\)}
    ]
    [\neg B \texttt{,} A \texttt{,} \neg C \texttt{,} B \texttt{,} \neg C, close
    ]
    ]
    [C \texttt{,} A \texttt{,} \neg C \texttt{,} B \texttt{,} \neg C, close]
    ]
    ]
    ]
    ]
    ]
  \end{tableau}
  \newline
\end{proof}


\section*{Problema 3.2}

Demostrar que si \(\vdash U\) en \(\mathcal{G}\) entonces existe un tableau semántico cerrado para \(\bar{U}\). 

\begin{proof}

  Sea \(T\) un tableau semántico (t.s.) para \(U\). Por inducción en la altura \(h\) de \(T\):\\

  Si \(h = 0\), entonces \(U\) consta de un solo par complementario de literales y, por lo tanto, también \(\bar{U}\). Por lo tanto, el tabelau semántico para \(\bar{U}\) es cerrado. \\

  Si \(h > 0\), entonces se aplicó alguna \(\alpha\)-fórmula o \(\beta\)-fórmula en la raíz de \(T\). Se sigue la demostración por casos.\\ 

  \begin{itemize}
    \item \textbf{Caso 1.} Se aplicó una \(\alpha\)-fórmula como \(A\) tal como \(A_1 \land A_2\) que corresponde a \(A_1, A_2\). Entonces, \(U = U_1 \cup U_2 \cup \{A\}\), donde \(\vdash U_1 \cup \{A_1\}\) y \(\vdash U_2 \cup \{A_2\}\). Por hipótesis de inducción, los complementos \(\bar{U_1} \cup \{\bar{A_1}\}\) y \(\bar{U_2}  \cup \{\bar{A_2}\}\) tienen tableaux cerrados. Por el teorema de robustez (\textit{soundness}) de t.s., estos conjuntos son insatisfactibles. Entonces, se sigue que sus superconjuntos \(\bar{U}' = \bar{U}_1 \cup \bar{U}_2 \cup \{A_1\}\) y \(\bar{U}'' = \bar{U}_1 \cup \bar{U}_2 \cup \{A_2\}\) también son insatisfactibles. Si tomamos \(\bar{A} = \bar{A_1} \lor \bar{A_2}\), podemos ver que constituye una \(\beta\)-fórmula. Por regla de tableau, \(\bar{U}\) se expande en dos hojas \(\bar{U}'\) y \(\bar{U}''\). Dado que ambas hojas son insatisfactibles y tienen tableaux cerrados, \(\bar{U}\) también los tiene.\\
    \item \textbf{Caso 2.} Se aplicó una  \(\beta\)-fórmula \(B\) tal como \(B_1 \lor B_2\) que corresponde a \(B_1,B_2\). Entonces, \(U = U_1 \cup \{B\}\) donde \(\vdash U' = U_1 \cup \{B_1, B_2\} \). Por hipótesis de inducción, el complemento \(\bar{U}' = \bar{U}_1 \cup \{B_1, B_2\}\) tiene un tableau cerrado. Tomando \(\bar{B} = \bar{B_1} \land \bar{B_2}\), vemos que corrrsponde a una \(\alpha\)-fórmula. Por regla de tableau, se expande \(\bar{U}\) a \(\bar{U}'\). Dado que \(\bar{U}'\) tiene un tableau cerrado, \(\bar{U}\) también lo tiene.
  \end{itemize}
  
\end{proof}

\section*{Problema 3.9}

Demostrar en  \(\mathcal{H}\):

\begin{align*}
    \vdash A & \rightarrow  A \lor B, \\
    \vdash B & \rightarrow A \lor B, \\
    \vdash (A \rightarrow B) & \rightarrow ((C \lor A) \rightarrow (C \lor B))
\end{align*}

\begin{proof}
  Por reglas de Hilbert:\\
  \noindent

  \begin{align*}
    1. &  & \{A \rightarrow B, \neg C \rightarrow A, \neg C \} &\vdash \neg C \rightarrow A & & \text{Suposición}\\
    2. & & \{A \rightarrow B, \neg C \rightarrow A, \neg C \} &\vdash \neg C & & \text{Suposición}\\ 
    3. & & \{A \rightarrow B, \neg C \rightarrow A, \neg C \} &\vdash A & & \text{M.P. 1,2}\\
    4. & & \{A \rightarrow B, \neg C \rightarrow A, \neg C \} &\vdash A \rightarrow B & & \text{Suposición}\\
    5. & & \{A \rightarrow B, \neg C \rightarrow A, \neg C \} &\vdash B & & \text{M.P. 3,4}\\
    6. & & \{A \rightarrow B, \neg C \rightarrow A\} &\vdash \neg C \rightarrow B & & \text{Deducción en 5}\\
    7. & & \{A \rightarrow B\} &\vdash (C \lor A) \rightarrow (C \lor B) & & \text{Deducción en 6}\\
  \end{align*}
\end{proof}

\section*{Problema 3.12}

Demostrar que el axioma 2 de \(\mathcal{H}\) es válido construyendo un tableau semántico para su negación.

\begin{proof}
  Por tableau semántico en la negación de la fórmula:\\
  \noindent

  \begin{tableau}
    {
      line no sep= 0.5cm,
      just sep= 0.5cm,
      for tree={s sep'=1mm},
      just refs right, % Set where crossreferences go
    }
    [\neg((A \rightarrow (B \rightarrow C)) \rightarrow ((A \rightarrow B) \rightarrow (A \rightarrow C))), just={Negación}, name=Prem
    [A \rightarrow (B \rightarrow C) \text{,} \neg ((A \rightarrow B) \rightarrow (A \rightarrow C)), just={\(\alpha \rightarrow\) }, name=NegConc
    [A \rightarrow (B \rightarrow C) \text{,} A \rightarrow B \text{,} \\ \neg(A \rightarrow C), just={\(\alpha \rightarrow\)}
    [A \rightarrow (B \rightarrow C) \text{,} A \rightarrow B \text{,} A \text{,} \neg C, just={\(\alpha \rightarrow\)},name=Alice
    [\neg A \text{,} A \rightarrow (B \rightarrow C) \text{,} A\text{,} \neg C , just={\(\beta \rightarrow\)},name=Bertie
    [\neg A \text{,} \neg A  \text{,} A  \text{,} \neg C, close, just={\(\beta \rightarrow\)}
    ]
    [B \rightarrow C \text{,} \neg A  \text{,} A  \text{,} \neg C
    [\neg B \text{,} \neg A \text{,} A \text{,} \neg C, close, just={\(\beta \rightarrow\)}
    ]
    [C \text{,} \neg A \text{,} A \text{,} \neg C, close
    ]
    ]
    ]
    [B \text{,} A \rightarrow (B \rightarrow C) \text{,} A\text{,} \neg C
    [B \text{,} \neg A \text{,} A \text{,} \neg C , close , just={\(\beta \rightarrow\)},name=Bertiex
    ]
    [B \rightarrow C \text{,} \neg A \text{,} A \text{,} \neg C
    [\neg B \text{,} B \text{,} A \text{,} \neg C, close, just={\(\beta \rightarrow\)}
    ]
    [C \text{,} B \text{,} A \text{,} \neg C, close
    ]
    ]
    ]
    ]
    ]
    ]
    ]
  \end{tableau}
  \newline
\end{proof}

\end{document}